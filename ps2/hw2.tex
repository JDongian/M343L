\documentclass{article}
\usepackage{amssymb}
\usepackage{amsmath}
\usepackage{centernot}
\usepackage{scalerel}
\usepackage{stackengine}
\usepackage{xcolor}
\newcommand\showdiv[1]{\overline{\smash{\hstretch{.5}{)}\mkern-3.2mu\hstretch{.5}{)}}#1}}
\newcommand\ph[1]{\textcolor{white}{#1}}


\makeatletter
% we use \prefix@<level> only if it is defined
\renewcommand{\@seccntformat}[1]{%
  \ifcsname prefix@#1\endcsname
    \csname prefix@#1\endcsname
  \else
    \csname the#1\endcsname\quad
  \fi}
% define \prefix@section
\newcommand\prefix@section{}
\newcommand\prefix@subsection{}
\makeatother

\begin{document}

\title{Applied Number Theory: Homework 1}
\author{Joshua Dong}
\date{\today}
\maketitle

\section{3}
\subsection{a)}
The subgroups of $(\mathbb{Z}/5, +)$ are
$(\mathbb{Z}/n, +)$ for all $n \in \{1, 2, 3, 4, 5\}$.
The subgroups of $(\mathbb{Z}/10, +)$ are
$(\mathbb{Z}/n, +)$ for all $n \in \{1, 2, 3, 4, 5, 6, 7, 8, 9, 10\}$.
\\\\
Assume $m$ is an integer. Then any element of $m\mathbb{Z}$ is an integer,
since integer multiplication is closed. Then $m\mathbb{Z} \subset \mathbb{Z}$.
\\*
Let $a_1, a_2 \in m\mathbb{Z}$.
Then there exist $k_1, k_2$ such that $a_1 = mk_1, a_2 = mk_2$.
Then $a_1 + a_2 = mk_1 + mk_2 = m(k_1 + k_2)$, by the distributive property
of $\mathbb{Z}$ on $(\cdot), +$. Since $\mathbb{Z}$ is closed under addition,
$(k_1 + k_2) \in \mathbb{Z}$ and there exists $k \in \mathbb{Z}$ such that
$a_1 + a_2 = mk \in m\mathbb{Z}$. Then $m\mathbb{Z}$ is closed under addition.
\\*
Since every element of $m\mathbb{Z}$ is an integer, we know that associativity holds in $m\mathbb{Z}$.
\\*
Let $a \in m\mathbb{Z}$. 
Then there exists $k$ such that $a = mk$.
Let $b = m(-k)$. $-k \in \mathbb{Z}$ since $\mathbb{Z}$ is a group.
Thus $m(-k) \in m\mathbb{Z}$.
Then $a + b = mk + m(-k) = mk + -(mk)$,
since $(\mathbb{Z}, \cdot)$ is an abelian group.
$mk + -(mk) = 0$.
Then $a + b = 0$, $b$ is an inverse of $a$.
Then for any $a \in m\mathbb{Z}$, there is an inverse also in $m\mathbb{Z}$.
\\*
Since we found an inverse for any element of $m\mathbb{Z}$,
a subset of the integers, we know that the identity element is unique since
we also showed the set is closed under the group operation.
\\*
Then $m\mathbb{Z}$ is a subgroup of $\mathbb{Z}$.

\subsection{b)}
i) The additive cosets of the subgroups $m\mathbb{Z}$ are the numbers of the form $g + mk$ where $g \in \mathbb{Z}$ and $k$ is an arbitrary integer.
Since $\mathbb{Z}$ is abelian, the left and right are the same (they are all normal).
\\\\
ii) The coset $gH$ is a subgroup of $G$ only when $g$ is the identity or the order of H divides $g$.
\\\\
iii) Take the set of all cosets of $H$, denoted $V$.
0, the identity, is in $H$. Then for all $g \in G$,
there exists an element $v \in V$ containing $g$, since
$g + 0 = g$.
\\\\
iv) We can create a mapping of any coset of H to any other coset of H.
If $g_1H, g_2H$ are a cosets of $H$, then we create the map
$f:g_1H \rightarrow g_2H$ where $f(x) = x + (-g_1) + g_2$.
This bijection (since we can trivially create $f^{-1}$ which is also injective and surjective) shows that any coset is the same size as any other coset.
\\\\
v) Suppose $g_1 = g_2$. Then trivially, $g_1H = g_2H$. Suppose
$g_1 \neq g_2$.
\\*
Suppose $x$ is in $g_1H$.
Then under the equivalence classes of all the left cosets of $H$, $x$ is in only one of them. Why? If $x$ were in two of them, then for the $g_a, g_b$ left cosets $x$ is in,
there would be elements $h_1, h_2 \in H$ such that $g_b + h_2 = x = g_a + h_1$.
This implies that $g_b = g_a + (h_1 + (-h_2))$, by the inverse and associative properties of groups. Then clearly, $g_b$ is congruent to $g_a$ under the $H$ coset equivalence class.

\subsection{c)}
Let the order of a subgroup $H$ be denoted $o(H)$, equal to the number of elements in $gH$, a left coset. Then $o(G)$ is the sum of $o(H)$, disjoint sets, $o(G) = sum for all unique cosets(o(H))$. Then clearly the order of a subgroup of $H$ divides the order of $H$.

\subsection{d)}
Fermat's little theorem concerns a mult. group, so the euler's toitient function, or $p - 1$ when $p$ is prime, as in this case, forms a multiplicitive ring with $p-2$ elements. As for any $a$ s.t. $a^{p-1} = a^{p-2} \cdot a$, $p-2$ is the order, so $a$ to that power is 1. then 1 times a is clearly a, under the group. formally, a to the power p-2 is equal to a to the power of some $xy$, integers such that $x$ is the order of a group, thus a to the power x is 1$

\section{1.36}
$X \equiv b \mod p$, $p \nmid b$.

\section{2.10}

\end{document}
