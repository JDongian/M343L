\documentclass{article}
\usepackage{amssymb}
\usepackage{amsmath}
\usepackage{centernot}
\usepackage{scalerel}
\usepackage{stackengine}
\usepackage{xcolor}
\newcommand\showdiv[1]{\overline{\smash{\hstretch{.5}{)}\mkern-3.2mu\hstretch{.5}{)}}#1}}
\newcommand\ph[1]{\textcolor{white}{#1}}


\makeatletter
% we use \prefix@<level> only if it is defined
\renewcommand{\@seccntformat}[1]{%
  \ifcsname prefix@#1\endcsname
    \csname prefix@#1\endcsname
  \else
    \csname the#1\endcsname\quad
  \fi}
% define \prefix@section
\newcommand\prefix@section{}
\newcommand\prefix@subsection{}
\makeatother

\begin{document}

\title{Applied Number Theory: Homework 1}
\author{Joshua Dong}
\date{\today}
\maketitle

\section{l.14}
\subsection{a)}
$b$ is positive and non-zero.
If $a$ is positive, we can choose $q = 0$, and $a$ will be in the set and we are done.
\\
If $a$ is 0, we can choose $q = -1$, and $b = 0 - b(-1)$ which will be in the set and we are done.
\\
If $a$ is negative, then we can choose
$q = a + 1$.
Then $a - b(a - 1) = a - ab + b = a(1 - b) + b$.
$b > 0$ implies $b - 1 > -1$, $1 - b < 1$.
If $1 - b$ is zero, $a(1 - b) + b = a(0) + b = b$, which is positive.
If $1 - b$ is negative (which is the only other case, since b is an integer), $a(1 - b)$ is positive since the product of two negative numbers is positive.
The sum of two positive numbers is positive, so $a(1 - b) + b$ would be positive.
\\
In all cases, there must be a positive element of the set.

\subsection{b)}
Consider the set of integer multiples of $b$ less than or equal to $a$. Denote this new set $A$. By the well-ordering principle, there is a greatest element of the set. Let this element be $xb$, where $x$ is an integer. 
\\
$a = xb + y$ for some $y \ge 0$.
$(x+1)b$ is not in the set $A$ but is a multiple, so $(x+1)b > a$.
$(x+1)b = xb + b > a = xb + y$.
$xb + b > xb + y$.
$b > y$.
\\
Then there exists $0 \le y < b$ for any integer choice of a and b.

\subsection{c)}
We already showed this in b) with the set of integer multiples. Simply take $q = x$, $r = y$.

\subsection{d)}
$a = bq_1 + r_1 = bq_2 + r_2$.
\\*
$bq_1 + r_1 = bq_2 + r_2$
\\*
$bq_1 - bq_2 = r_2 - r_1$
\\*
$b(q_1 - q_2) = r_2 - r_1$.
\\*
But this means the difference between $r_1$ and $r_2$ is divisible by b. We know the two remainders are greater than 0. If their difference was greater than 0, then their difference would be at least $b$ (by the definition of divisibility, and since $q$ is integral). This is a contradiction with the assumption that $r_{1,2} < b$.
\\
Then their difference must be 0, the only remaining possibility. If the remainders are the same, then $q_1$ and $q_2$ must be the same by the assertion of their relation to $a$.


\section{l.23}
$m$ is odd. $a$ is an integer.
First we show that squares of even integers are equal to 0 modulo 4, and squares of odd integers are equal to 1 modulo 4.
\\
Let $2k$ be an even integer, for some integer k, by definition of even.
Then $(2k)^2 = 4k^2 = 4(k^2)$ is equivalent to 0 modulo 4 by definiton of modulo.
Let $2k + 1$ be an even integer, for some integer k, by definition of odd.
Then $(2k + 1)^2 = 4k^2 + 4k + 1 = 4(k^2 + k) + 1$ is equivalent to 1 modulo 4 by definiton of modulo.
\\
Since $m$ is odd, by definition there exists some integer $x$ such that $m = 2x + 1$.
Then $2m + a^2 \equiv
2(2x +1) + a^2 \equiv
4x + 2 + a^2 \equiv
2 + a^2
\mod{4}$.
\\
If a is even, we showed its square is equivalent to 0 modulo 4. Then $2 + a^2 \equiv 2 \mod{4}$. This cannot be a square, as shown previously.
\\
If a is odd, we showed its square is equivalent to 1 modulo 4. Then $2 + a^2 \equiv 3 \mod{4}$. This cannot be a square, as shown previously.
\\
Then any number in the form of $2m + a^2$ where m is odd can never be a perfect square.


\section{l.25}
This is the same algorithem I implemented for fast modular exponentiation. It works since we take the binary representation and multiply the exponents of those bases. The floor function is simply a bit shift, taking the next least significant binary digit. This loop uses b to save the value of $g^{2^{loop iteration}}$.
\\\\\\
sorry I typed these in a hurry!

\end{document}
