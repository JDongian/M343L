\documentclass{article}
\usepackage{amssymb}
\usepackage{fullpage}
\usepackage{amsmath}
\usepackage{centernot}
\usepackage{scalerel}
\usepackage{listings}
\usepackage{color}
\usepackage{xcolor}
\usepackage{graphicx}
\usepackage{algorithm}
\usepackage{algpseudocode}
\usepackage{pifont}
\usepackage{parskip}
\usepackage[english]{babel}
\usepackage{biblatex}
\usepackage{hyperref}
\addbibresource{citations.bib}

\newcommand{\vars}{\texttt}
\newcommand{\func}{\textrm}
\let\oldReturn\Return
\renewcommand{\Return}{\State\oldReturn}


\definecolor{mygreen}{rgb}{0,0.6,0}
\definecolor{mygray}{rgb}{0.5,0.5,0.5}
\definecolor{mymauve}{rgb}{0.58,0,0.82}

\lstset{ %
  backgroundcolor=\color{white},   % choose the background color; you must add \usepackage{color} or \usepackage{xcolor}; should come as last argument
  basicstyle=\ttfamily,        % the size of the fonts that are used for the code
  breakatwhitespace=false,         % sets if automatic breaks should only happen at whitespace
  breaklines=true,                 % sets automatic line breaking
  captionpos=b,                    % sets the caption-position to bottom
  commentstyle=\color{mygreen},    % comment style
  deletekeywords={...},            % if you want to delete keywords from the given language
  escapeinside={\%*}{*)},          % if you want to add LaTeX within your code
  extendedchars=true,              % lets you use non-ASCII characters; for 8-bits encodings only, does not work with UTF-8
  frame=single,	                   % adds a frame around the code
  keepspaces=true,                 % keeps spaces in text, useful for keeping indentation of code (possibly needs columns=flexible)
  keywordstyle=\color{blue},       % keyword style
  language=C,                 % the language of the code
  morekeywords={*,...},           % if you want to add more keywords to the set
  rulecolor=\color{black},         % if not set, the frame-color may be changed on line-breaks within not-black text (e.g. comments (green here))
  showspaces=false,                % show spaces everywhere adding particular underscores; it overrides 'showstringspaces'
  showstringspaces=false,          % underline spaces within strings only
  showtabs=false,                  % show tabs within strings adding particular underscores
  stringstyle=\color{mymauve},     % string literal style
  tabsize=2,	                   % sets default tabsize to 2 spaces
  title=\lstname                   % show the filename of files included with \lstinputlisting; also try caption instead of title
}


\begin{document}

\title{Attacks on Elliptic Curve Cryptography}
\author{Joshua Dong}
\date{\today}
\maketitle

%\section{Abstract}
\begin{abstract}
    Assumptions on the hardness of problems such as the discrete logarithm
    problem and integer factorization are challenged by advances in modern
    cryptography and computing. Advances in quantum computing cast significant
    doubt on the security of currently used algorithms such as Diffie-Hellman,
    ElGamal, and RSA which rely on these assumptions of hardness. Growing
    discussion on elliptic curve cryptography is promising, but this paper
    looks to challenge and qualify the merits of elliptic curve cryptography.
    With many pitfalls in theory and implementation, elliptic curve protocols
    are reviewed with a focus on mitigating known attacks.
\end{abstract}
 

\section{Background}
\subsection{Hyperelliptic Curves}
    A \textit{hyperelliptic curve} is a plane curve of the form
    \begin{equation}
        y^2 = f(x),
    \end{equation}
    where the $n$ degree polynomial $f(x)$ has $n$ distinct roots. Then the
    genus of such a curve is $g$ where $n = 2g + 1$ or $n = 2g + 2$. Elliptic
    curves are hyperelliptic curves of genus 1.

    One may wonder why elliptic curves of genus 1, and not conics or
    hyperelliptic curves of genus greater than 1 are used in cryptography. The
    answer lies in a balance between computational simplicity, group operation
    complexity, and smoothness properties of the curve used. Conic
    cryptography, while fast, adds no security benefit in terms of hardness in
    proportion to key size (and in the case of singular curves with a cusp have
    polynomial time solutions) \cite{Menezes1992}. The weaknesses of
    hyperelliptic curves of higher genus is discussed later.

\subsection{Edwards Curves}
    There are many forms of elliptic curves and many ways to represent them.
    Optimizing the group operations in finite fields over elliptic curves has
    been a major area of research. This section provides a brief review of the
    topics and notations relevant to this paper.

    The general Weierstrass form of an elliptic curve $E$ is:
    \begin{equation}
        y^2 + a_1xy + a_3y = x^3 + a_2x^2 + a_4x + a_6.
    \end{equation}
    $E(K)$ is the set of points $(x, y) \in K^2$ for some field $K$. Adding a
    "point at infinity" to serve as the identity element, the set forms an
    abelian group under point addition. When $K$ does not have characteristic 2
    or 3, $E$ may be represented in the more common form
    \begin{equation}
        y^2 = x^3 + ax + b
    \end{equation}
    where $a$ and $b$ are subject to constraints preventing cusps and
    self-intersections. Every elliptic curve over a finite field may be
    represented as an \textit{Edwards curve} \cite{tec} of the form 
    \begin{equation}
        x^2 + y^2 = c^2(1 + x^2y^2)
    \end{equation}
    These Edwards curves not only generalize elliptic curves, but also benefit
    from a unified addition law that is beneficial for computation:
    \begin{equation}
        (x_1, y_1) + (x_2, y_2) =
        (\frac{x_1y_2 + y_1x_2}{1 + dx_1x_2y_1y_2},
         \frac{y_1y_2 - x_1x_2}{1 - dx_1x_2y_1y_2}).
    \end{equation}
    The unified addition law is not only convenient, but is useful against
    simple power analysis attacks which exploit non-unified forms of the group
    operation as seen some other elliptic curves. The results of Bernstein,
    Birkner, Joye, Lange, and Peters deal with variants of Edwards curves over
    non-binary fields that improve efficiency of addition and doubling in the
    group.

    % TODO: homogeneous, Jacobian projective coordinate explanation

\subsection{Scalar Multiplication}
    A common hardness assumption in elliptic curve cryptography is the elliptic
    curve discrete logarithm problem. The idea is that for a point $P$ on an
    elliptic curve $E$, a secret integer $d$ is hard to determine only given
    $P$, $E$, and $dP$. There are many ways to multiply $P$ by $d$ in
    sub-linear time. Relevant multiplication algorithms are covered below.
 
    The most straightforward logarithmic implementation of scalar multiplication
    is a variant of the "square-and-multiply" algorithm used in RSA.

    \begin{algorithm}
        \caption{Double-and-Add from the most significant bit}
        \label{Double-and-Add}
        \begin{algorithmic}
            \Require $d, P$
            \Ensure $Q = dP$
            \State $Q := P$
            \For{$i = n - 2 \to 0$}
                \If{$d_i = 1$}
                    \State $Q := Q + P$
                \EndIf
            \EndFor
            \Return{Q}
        \end{algorithmic}
    \end{algorithm}

    This implementation is vulnerable to simple power analysis or timing attack
    due to the branch statement that causes uneven computations correlated to
    the bit pattern of the secret \cite{coron1999resistance}. Thus, Coron
    proposed the double-and-add-always method which always executes the
    addition and prevents this attack.

    \begin{algorithm}
        \caption{Double-and-Add-Always from the most significant bit}
        \label{Double-and-Add-Always}
        \begin{algorithmic}
            \Require $d, P$
            \Ensure $Q_0 = dP$
            \State $Q_0 := P$
            \For{$i = n - 2 \to 0$}
                \State $Q_0 := 2Q_0$
                \State $Q_1 := Q_0 + P$
                \State $Q_0 := Q_{d_i}$
            \EndFor
            \Return{$Q_0$}
        \end{algorithmic}
    \end{algorithm}

    This algorithm for scalar multiplication removes the computation difference
    between 0 and 1 bits.

    Also used for scalar multiplication is the sliding-window method, which is
    the same as the double-and-add algorithm but chooses a fixed window value
    $w$. It similarly does not give away information in the manner of the
    Double-and-Add algorithm.

    \begin{algorithm}
        \caption{Sliding-Window Method}
        \label{Sliding-Window Method}
        \begin{algorithmic}
            \Require $d, P$
            \Ensure $Q_0 = dP$
            \State $Q = 0$
            \For{$i = m \to 0$}
                \If{$d_i = 0$}
                    \State $Q := 2Q$
                \Else
                    \State $t := $ extract $min(j, w-1)$ additional bits from $d$
                    \State $i := i - j$
                    \If{$j < w$}
                        \State $Q := tQ$ 
                        \Return{Q}
                    \Else
                        \State $Q := 2^wQ$
                        \State $Q := Q + tP$
                    \EndIf
                \EndIf
            \EndFor
        \end{algorithmic}
    \end{algorithm}

    Montgomery originally proposed the Montgomery method for scalar
    multiplication in constant time \cite{montgomery1987speeding}. It is also
    resistant against simple power analysis \cite{okeya2000power}.

    \begin{algorithm}
        \caption{Montgomery Ladder}
        \label{Montgomery Ladder}
        \begin{algorithmic}
            \Require $d, P$
            \Ensure $Q_0 = dP$
            \State $Q_0 := P$
            \State $Q_1 := 2P$
            \For{$i = n - 2 \to 0$}
            \State $Q_{1 - d_i} := Q_0 + Q_1$
            \State $Q_{d_i} := 2Q_{d_i}$
            \EndFor
            \Return{$Q_0$}
        \end{algorithmic}
    \end{algorithm}

    These multiplication algorithms account for all major algorithms in use for
    scalar multiplication on elliptic curves. However, each of these algorithms
    have weaknesses to side-channel attacks, as explained later.

\subsection{Obfuscations}
    Presented below are some of the commonly used obfuscations on scalar
    multiplication to prevent side-channel attacks on elliptic curve discrete
    logarithm protocols. The details are covered in Goubin's work, but a
    summary is provided here for context \cite{goubin2003refined}.

\subsubsection{Random Projective Coordinates}
    Multiplication of a scalar $d$ and a point $P$ is performed in the
    projective coordinates. The base-point $P = (x, y)$ is represented as
    homogeneous or Jacobian projected coordinates on some random
    $\theta \in K$, where $\theta$ is invertible. Then $dP$ can be computed in
    the projective space on $\theta$, and then transformed back into affine
    coordinates using $\theta^{-1}$.

\subsubsection{Random Elliptic Curve Isomorphisms}
    For some elliptic curve $E: y^2 = x^3 + ax + b$ on a field $K$ of
    characteristic not 2 or 3, scalar multiplication is done by choosing a
    random $\theta \in K$ and computing using homogeneous projective
    coordinates $dP'$, where $P' = (\theta^2x, \theta^3y, 1)$ in
    $E': Y^2Z = X^3 + \theta^{−4}XZ^2 + \theta^{−6}bZ^3$.

\subsubsection{Random Field Isomorphisms}
    An elliptic curve $E$ is applied over a Galois field of order $2^m$,
    $GF(2^m)$. Then the field can also be represented as the quotient group
    $GF(2)[x]/f(x)$, where $f(x)$ is an irreducible polynomial of degree $m$
    over $GF(2)$. Since many such polynomials exist, the idea is that many
    isomorphic fields can be created easily. Since all these finite fields are
    still isomorphic to $GF(2^m)P$, an inverse function (of the isomorphism)
    can be applied after multiplication in the image of the isomorphism in $E$.

\subsection{Supersingular Isogenies}
    Quantum cryptography provides effective solutions to the discrete logarithm
    for elliptic curves. This is because the discrete logarithm can be
    efficiently computed over arbitrary fields using Shor's algorithm
    \cite{shor}.  Fortunately, there are more hard problems in elliptic curve
    cryptography such as problems concerning supersingular isogenies, which are
    still hard for quantum computers to solve. We introduce the context of
    supersingular elliptic curves and isogenies here, but the application to
    security is discussed later.

    Suppose we have two elliptic curves $E_1$ and $E_2$ over a finite field
    $\mathbb{F}_q$. A rational mapping between two algebraic varieties (such as
    elliptic curves) is the data of a partial function
    $f: E_1 \supset U \rightarrow X$ modulo the equivalence relation
    $f_1: E_1 \supset U_1 \rightarrow X \equiv
    f_2: E_1 \supset U_2 \rightarrow X$ if there exists $W \subset U_1 \cap U_2$
    where $f_1$ and $f_2$ agree on $W$ \cite{class13}. An \textit{isogeny}
    $\varphi: E_1 \rightarrow E_2$ over $\mathbb{F}_q$
    is a rational mapping that is also a group homomorphism from
    $E_1 (\mathbb{F}_q)$ to $E_2 (\mathbb{F}_q)$. Two elliptic curves
    $E_1$ and $E_2$ defined over $\mathbb{F}_q$ are \textit{isogenous} over
    $\mathbb{F}_q$ if there exists an isogeny between them.

    An \textit{endomorphism} of an elliptic curve $E$ defined over
    $\mathbb{F}_q$ is a self-homomorphism isogeny defined over
    $\mathbb{F}_{q^m}$ for some $m$. The set of endomorphisms of E together
    with the zero map is a ring under pointwise addition and function
    composition; this ring is called the \textit{endomorphism ring} of $E$ and
    denoted $End(E)$ \cite{jao2011towards}.  When the ring End(E) is isomorphic
    to an order in a quaternion algebra we say that $E$ is
    \textit{supersingular}. That is, the number of basis vectors ($dim(Im(T))$)
    of the endomorphism transformation is the rank of the endomorphism, which
    is always 4 for supersingular elliptic curves.

    The curves over a field $\mathbb{F}_{q}$ with equivalence under isomorphism
    in $\overline{\mathbb{F}}_{q}$ form an \textit{isogeny graph}. Less
    formally, it is a graph with nodes representing the equivalence classes of
    elliptic curves under isomorphism.  Note that j-invariants are used to
    represent these equivalence classes, easy to compute from a given elliptic
    curve $y^2 = x^3 + px + q$:
    \begin{equation}
        j = 1728 \frac{4p^3}{4p^3 + 27q^2}.
    \end{equation}

\section{Attacks}
    This section will look towards the exploitation of weaknesses in elliptic
    curve cryptography. There are both theoretical and implementation issues
    that arise when building a crypto-system, and elliptic curve cryptography
    is no exception.  Attention to implementing a cryptographic protocol is
    important, as events such as the introduction of \textsc{Dual\_EC\_DRBG}
    emphasize. While the topic of back-doors will not be discussed here, we
    will look at standard mathematical and side-channel attacks on currently
    existing implementations of elliptic curve cryptography.
    
\subsection{Mathematical Attacks}
    Insecurity of a protocol can be shown with an algorithm that compromises
    the hardness assumptions of that protocol. Security is much harder to
    prove, as assumptions of hardness are but assumptions. In the following
    sections, some of the major attempts to find weaknesses in elliptic curve
    cryptography will be discussed. This paper does not cover the MOV attack in
    detail, an attack that reduces the elliptic curve discrete logarithm
    problems into the discrete logarithm over multiplicative integer groups by
    creating a bilinear pairing from $E(\mathbb{F}_q)$ into $\mathbb{F}_{q^k}$.
    The MOV attack can be mitigated by choosing curve parameters with a large
    embedding degree $k$.

\subsubsection{Index Calculus}
    Index calculus algorithmically computes discrete logarithms of finite
    integer groups with prime order. However, the probabilistic algorithm may
    be applied to other groups. Attempts to use index calculus on elliptic
    curves have failed due to the differences between the smoothness of finite
    fields on elliptic curves and integers modulo some $n$.

    Index calculus algorithms rule out the usage of hyperelliptic curves of
    genus greater than or equal to 3, because of the same smoothness
    characteristics that make multiplicative groups weak to index calculus
    \cite{Theriault2003} \cite{Gaudry2000}. Hyperelliptic curves of increasing
    genus have increasingly hard-to-compute group operations, and hyperelliptic
    curves of genus 1 have the fast group operations \cite{Costello2012}. For
    both these reasons, hyperelliptic curves of genus 1 are the focus of
    cryptographic protocols.

    Since prime elements in elliptic curves do not have the same properties as
    primes in multiplicative groups, it is impossible to find an effective
    factor base for index calculus. The reason for this is that for index
    calculus in the multiplicative group, if $r$ is the rank of the chosen
    factor base, then $p_1, p_2, ..., p_r$, the primes in the factor base, are
    small such that
    \begin{equation}
        \log{p_i} \approx \log{i} \le \log{r}.
    \end{equation}
    However, when choosing a rank $r$ for elliptic curves, the prime elements
    of the factor base are on the order of $r\log{r}$ \cite{Silverman1998}.
    This means that there will be no rank $r$ such that both computation is
    easy (sub-exponential) and the probability that a lifted candidate from the
    index calculus algorithm is $p_r$-smooth is high. Choosing a small $r$ for
    sub-exponential computation never results in a decent collision
    probability, since the primes of elliptic curve groups are too large. At
    the same time, choosing a large $r$ defeats the purpose of index calculus,
    as the system of equations (the computation bottleneck) to solve becomes
    proportionately large.  Additionally, we have assumed that the during the
    lifting step of index calculus, the scalar multiplication was done into the
    global field $\mathbb{Q}$.  Other lifting strategies have been proposed,
    none of which are effective \cite{Silverman2009}. Thus, index calculus is
    not particularly effective for solving elliptic curve discrete logarithms. 

%\subsubsection{MOV Attack}
%A lesser known attack on elliptic curves is the MOV attack. The attack 

\subsubsection{Quantum attacks}
    It is well known that integer factorization can be solved in
    sub-exponential time using quantum computing \cite{quant_fac}
    \cite{shor_fac}. Thus concerns have arisen over the use of RSA and similar
    algorithms which rely on the hardness of non-quantum integer factorization.
    However, other previously considered difficult problems such as the
    discrete logarithm problem also become solvable in sub-exponential time as
    a consequence of quantum computers \cite{bern_quant}.

    While Shor's algorithm provides a means for sub-exponential integer
    factorization and discrete logarithms, it also provides a sub-exponential
    solution to the elliptic discrete logarithm problem. This is significant
    because fewer $q$-bits are needed in elliptic curve cryptography than for
    equivalent-hardness problems of integer factorization or the discrete
    logarithm over finite fields. In fact, the bottleneck in reducing the
    number of $q$-bits needed for solving elliptic discrete logarithms is in
    the use of the extended Euclidian algorithm during division, not modular
    exponentiation \cite{proos}.  Thus, breaking 224-bit elliptic curve
    cryptography (equivalent to 2048-bit RSA) would require less than 1600
    $q$-bits. To put this in perspective, the most powerful universal quantum
    computers of 2016 only had 5 $q$-bits of computing power \cite{ibm}. While
    the future of quantum computing is far from certain, the theoretical risks
    are strong motivator to move to quantum-resistant hardness assumptions.

    Supersingular isogeny Diffie-Hellman (SIDH) key exchange, however, relies
    on finding the isogeny mapping between two supersingular elliptic curves
    with the same number of points. Recall that an isogeny is a surjective
    mapping between curves that preserves the base points (generators) and
    structure (homomorphic). For elliptic curves with endomorphism rank not
    equal to 4, the endomorphism ring is commutative, and ideal classes form a
    finite abelian group. This property, along with the use of a special group
    action, turns the problem of finding an isogeny into the abelian group
    hidden shift problem, which can be solved in quantum sub-exponential time
    \cite{childs2007quantum} \cite{childs2014constructing}. As quaternion
    algebra in not commutative, no group action exists for supersingular
    elliptic curves and this problem reduction is not possible. The hardness of
    finding supersingular isogenies is the basis of SIDH, where public keys are
    supersingular curves and private keys are the isogenies. SIDH shared keys
    are produced by applying a private isogeny to a public curve, creating
    curves of equal j-invariant (isomorphic, thus equivalent under the isogeny
    graph) \cite{jao2011towards}. Feo, Jao, and Plut provide a detailed
    explanation.

\subsection{Side-channel attacks}
    A side-channel attack is an attack which monitors a source of side-channel
    data leakage to infer sensitive data. Electro-magnetic radiation, current
    usage, and timing differences are all examples of side-channels through
    which data may be leaked. A motivating factor behind elliptic curve
    cryptography adoption is the smaller key sizes, as key compromise is easier
    with larger keys \cite{walter2003longer} \cite{yarom2014recovering}. While
    longer key sizes provide more mathematical security, the security of
    practical implementations generally suffer due to an increased attack
    surface with respect to side-channel data leaks.

    For security protocols relying on the elliptic curve discrete logarithm,
    the most sensitive step of computation is the scalar multiplication of a
    point $P$ by a factor $n$. Whatever the method used to multiply, if every
    operation of the processor were known, then the key would be easily
    extracted. However, operations or branches are only inferred by increases
    in power or time usage.  This section investigates the some security
    concerns of elliptic curve cryptographic implementations currently in use,
    all of which rely on the elliptic curve discrete logarithm.

\subsubsection{Differential Power Analysis}
    Differential power analysis is the interpretation of electrical usage over
    time using statistics. The attack exploits biases in power consumption of
    hardware during cryptographic computations. Incorporating signal processing
    methods, differential power analysis can extract secrets from measurements
    which contain random or systematic noise. 

    Random projective coordinates, random elliptic curve isomorphisms, and
    random field isomorphisms, are three common methods of obfuscating against
    power analysis attacks have be briefly covered above. This is because the
    introduced randomness denies an attacker knowledge of the internal state,
    given a random enough parameter. The natural approach to thwarting these
    measures uses statistics to attempt to "even out" the random noise.
    Differential power analysis can indeed thwart these obfuscation attempts,
    even when used in conjunction with even-power usage multiplications.

    For example, when using Double-and-Add-Always to mitigate side-channel data
    leakage, two sets of inputs can be fed to the multiplier to find the value
    of a bit with index $i$ (given prior knowledge of the bits before $i$).
    The two sets of inputs may be associated with some feature of the output in
    such a way that distinguishes the power consumption curves using
    correlation with the assigned set of the input. More formally, for an $n+1$
    bit secret $d = (d_{n-1}, d_{n-2}, ..., d_0)$ and prior knowledge of bits
    $d_{n-1}, ..., d_{j+1}$, an attacker can guess that $d_j$ is 1
    and choose arbitrary points
    $P_0, P_1, ..., P_t$. Then the attacker computes
    $Q_r = \sum_{i=j}^{n-1} k_i2^{i}P_r$
    for every possible $P_r$.
    Then the attacker divides the set $\{P_0, P_1, ..., P_t\}$ into two sets
    $A = \{A_0, ...\}$ and $B = \{B_0, ...\}$ such that
    $P_i \in A$ if $\tau(Q_i) = true$, else $P_i \in B$, where $\tau$ is a
    boolean function dependent on its parameter but also splitting elements
    somewhat evenly between $A$ and $B$ ($|A| \neq 0, |B| \neq 0$).  Encrypting
    the values $P_0, P_1, ..., P_t$ and observing the current, differences
    between the respective side-channel signals for the sets $A$ and $B$ will
    be near 0 if the guess is incorrect (no correlation with set assignment)
    and non-zero if the guess is correct (non-zero correlation with set
    assignment).  Specifically, correlation between the signal $\phi(r)$
    associated with computing $kP_r$ can be measured with
    \begin{equation}
        Average_{P_r \in A}(\phi(r)) - Average_{P_r \in B}(\phi(r)).
    \end{equation}
    After the value of $d_j$ is determined, the same strategy can be applied
    recursively to determine the entire secret. This method will work for any
    side-channel, but fails when random transformations are used (such as
    random projective coordinates) \cite{joye2003elliptic}.
   
    To tackle implementations using both even-power scalar multiplication as
    well as randomization (via projection, group isomorphism, or curve
    isomorphism), a similar approach can be used. Since even under random
    isomorphism or random point projection, projective coordinate zeros are
    still zero, a special point $P' = (X, Y, Z)$ not equal to the identity with
    $X=0$ or $Y=0$ or $Z=0$ can be used to determine the secret key. The
    details of attacking every combination of obfuscation method with every
    scalar multiplication algorithm are detailed by Goubin
    \cite{goubin2003refined}.
    

\subsubsection{\sc{Flush+Reload}}
    The \textsc{Flush+Reload} attack \cite{yarom_falkner} is a specialized timing
    attack exploiting cache optimizations on Intel x86 processors.  It operates
    by differentiating cached and non-cached memory using timing differences. A
    spy program periodically flushes the memory line from the processor cache,
    then after a short delay, loads data from memory. Since the memory line is
    flushed at the beginning, the spy program will know whether the cached
    instructions were used by measuring the access time of memory reads. The
    attack is mitigated by replacing branch instructions or implementing access
    controls to memory pages.

    The Elliptic Curve Digital Signature Algorithm implemented in the OpenSSL
    libraries before version 1.0.0m were vulnerable to the \textsc{Flush+Reload}
    attack. Old versions used a na\"{i}ve implementation of the Montgomery ladder for
    scalar multiplication:
    \newpage
    \lstinputlisting{ec2_mult_old.c}
    In this code, the first \texttt{if} statement can be monitored by a
    \textsc{Flush+Reload} spy program to infer each bit of \texttt{word}, the
    secret key. Details of how this was executed are explained by Yarom and
    Benger \cite{yarom2014recovering}.

    The branching issue was mediated with a constant-time conditional swap.
    This implementation does not leak data because the memory lines do not hold
    differentiating instructions.
    \lstinputlisting{ec2_mult_new.c}

    Work has also produced similar success on the implementation of
    \textbf{secp256k1} in OpenSSL for sliding-window multiplication
    \cite{benger2014ooh}. However, the attack described is more complicated as
    the partial knowledge of point addition and doubling sequences is combined
    with a lattice reduction approach to determine the 256 bit key.


\section{Conclusion}
Research on elliptic curve curve cryptography is still ongoing. The major
selling point of elliptic curve cryptographic protocols is a much smaller key
size. Computations are more expensive than equivalent-security algorithms using
integer factorization or the multiplicative discrete logarithm problem.
However, a smaller key size offers significant security benefits against side
channel attacks. If quantum computing is to pose a real challenge to current
cryptographic protocols, then those protocols built on the elliptic curve
discrete logarithm will be the first to become ineffective. Other hard problems
exist that build on elliptic curve theory, such as the problem of finding an
isogeny mapping between supersingular elliptic curves -- which turns out to be
resistant to quantum attacks. Even in a world without powerful quantum
computers, care must be taken when implementing cryptographic protocols, since
side-channel attacks can take place invisibly, without contact. Elliptic curve
theory is promising for cryptography, but like other basis of security
assumptions, may only seem secure because fast algorithms have yet to be
discovered.


\printbibliography

\end{document}
