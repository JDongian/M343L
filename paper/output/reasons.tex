\documentclass{article}
\usepackage{amssymb}
\usepackage{amsmath}
\usepackage{centernot}
\usepackage{scalerel}
\usepackage{xcolor}
\usepackage{graphicx}
\usepackage{parskip}
\usepackage[english]{babel}
\usepackage{biblatex}
\addbibresource{citations.bib}

\newcommand\showdiv[1]{\overline{\smash{\hstretch{.5}{)}\mkern-3.2mu\hstretch{.5}{)}}#1}}
\newcommand\ph[1]{\textcolor{white}{#1}}

\makeatletter
% we use \prefix@<level> only if it is defined
\renewcommand{\@seccntformat}[1]{%
  \ifcsname prefix@#1\endcsname
    \csname prefix@#1\endcsname
  \else
    \csname the#1\endcsname\quad
  \fi}
% define \prefix@section
\newcommand\prefix@section{}
\newcommand\prefix@subsection{}
\makeatother

\begin{document}

\title{Concerns on ECC}
\author{Joshua Dong}
\date{\today}
\maketitle

\section{Abstract}
Assumptions on the hardness of problems such as the discrete logarithm problem
and integer factorization are challenged by advances in modern crytpograhpy.
Advances such as quantum cryptography cast significant doubt on the security of
currently used algorithms such as Diffie–Hellman, ElGamal, and RSA which rely
on these assumptions of hardness. Growing discussion on elliptic curve
cryptography is promising, but this paper looks to challenge and qualify the
merits of elliptic curve cryptography. While many pitfalls exist, there are
promising ideas in the field of ECC, some of which are introduced at the end of
the paper.
 

\section{Background}
\subsection{Elliptic Curve Cryptography}
There are many forms of elliptic curves and many ways to represent them.
Optimizing the group operations in finite fields over elliptic curves has been
a major area of research. This section provides a brief review of the topics
and notations relevant to this paper.

The Weierstrass form of elliptic curves is
\begin{equation}
    y^2 = x^3 + ax + b
\end{equation}
Where $a$ and $b$ are subject to constraints preventing cusps and
self-intersections. Every elliptic curve over a finite field may be represented
as an Edwards curve \cite{tec} of the form 
\begin{equation}
    x^2 + y^2 = c^2(1 + x^2y^2)
\end{equation}
These Edwards curves not only generalize elliptic curves, but also benefit from
a unified addition law that is beneficial for computation:
\begin{equation}
    (x_1, y_1) + (x_2, y_2) =
    (\frac{x_1y_2 + y_1x_2}{1 + dx_1x_2y_1y_2}, \frac{y_1y_2 - x_1x_2}{1 - dx_1x_2y_1y_2})
\end{equation}
The results of Bernstein, Birkner, Joye, Lange, and Peters deal with variants
of Edwards curves over non-binary fields that improve efficiency of addition
and doubling in the group.

One may wonder why elliptic curves of genus 1, and not conics or hyperelliptic
curves of genus greater than 1 are used in cryptography. The answer lies in a
balence between computational simplicity, group operation complexity, and
smoothness properties of the curve used. Conic cryptography, while fast, adds
no security benefit in terms of hardness in proportion to key size
\cite{Menezes1992}. Hyperelliptic curves of increasing genus have increasingly
hard-to-compute group operations, and hyperelliptic curves of genus 1 or 2 have
the fastest group operations. Additionally, curves of genus greater than or
equal to 4 have weaknesses in the smoothness of the group elements
\cite{Costello2012} \cite{Gaudry2000}. Specifically, index calculus becomes
very effective for higher genus curves (while still effective for elliptic
curves of low genus with poorly chosen parameters).


\subsection{Quantum Cryptography}
It is well known that integer factorization can be solved in sub-exponential
time using quantum computing \cite{quant_fac} \cite{shor_fac}. Thus concerns
have arisen over the use of RSA and similar algorithms which rely on the
hardness of non-quantum integer factorization. However, other previously
considered difficult problems such as the discrete logarithm problem also
become solvable in sub-exponential time as a consequence of quantum computers
\cite{bern_quant} \cite{shor}.

While Shor's algorithm provides a means for sub-exponential integer
factorization and discrete logarithms, it also provides a sub-exponential
solution to the elliptic discrete logarithm problem. This is significant
because fewer $q$-bits are needed in elliptic curve cryptography than for
equivalent-hardness problems of integer factorization or the discrete logarithm
over finite fields. In fact, the bottleneck in reducing the number of $q$-bits
needed for solving elliptic discrete logarithms is in the use of the extended
Euclidian algorithm during division, not modular exponentation \cite{proos}.
Thus, breaking 224-bit elliptic curve cryptography (equivalent to 2048-bit RSA)
would require less than 1600 qbits. To put this in perspective, private
companies already had quantum computers with over 1000 $q$-bits \cite{dwave} in
2015, and 1600 $q$-bit computers are predicted to exist by 2017 \cite{pq}.
While the future of quantum computing is far from certain, the theoretical
risks are strong motivator to move to quantum-resistant hardness assumptions.

Fortunately, there are more hard problems in elliptic curve cryptography such
as problems concerning supersingular isogenies, a topic which will be covered a
later section of this paper.

\section{Attacks}
still hard to implement, and a fiasco such as dual ec drbg can only be
prevented with vigilant inspriation
brief mention of does dual ec drbg have a place in this paper? (probably not)

\subsection{index calclusu}
finding a Weil pairing means (no subexp) ecdlp = dlp (subexp)

\subsection{MOV Attack}
finding a Weil pairing means (no subexp) ecdlp = dlp (subexp)

\subsection{Quantum attacks}
%ECDHE
parameter selection, hyperelliptic curves

\subsection{Side channel attacks}
\subsubsection{Simple Power Analysis}
[Huseyin Hisil, Kenneth Koon-Ho Wong, Gary Carter, Ed Dawson twisted ed curves revisited]
2M + 1D unified addition on a 4 core

\subsubsection{Timing attacks}
[Huseyin Hisil, Kenneth Koon-Ho Wong, Gary Carter, Ed Dawson twisted ed curves revisited]

\subsubsection{FLUSH+RELOAD}
montgomery cache attack (yarom benger flush reload), 
window attack (ohh ahh Naomi Benger 1 , Joop van de Pol 2 , Nigel P. Smart 2 , and Yuval Yarom 1)

\section{Promising Ideas}
SIDH, ed25519 if quantum doesn't work out (it's a lot of bits -- calc [proos zalka])w
However, if state agencies have access to technologies surpassing the private sector,
...

\section{Conclusion}
While ellptic curve crytpograhpy may be a better alternative to DHE or DLP, 
there are still problems -- mostly with implementation. The requisite knowledge
in abstract algebra to understand ECC hinders software developers from easily
accepting the use of ECC. Addtionally, ECDLP offers little protection from 
large quantum computers. Even in a future without powerful >1000qbit computers,
must be careful to set proper parameters just as integer factorization relied
on a good choice of primes. ed25516 may be the best choice in this case. However,
most ECC ideas have their flaws. SIDH presents a promising direction in the
development of ECC, but perhaps it is only because major flaws have yet to be revealed.


\printbibliography

\end{document}
