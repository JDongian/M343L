\documentclass{article}
\usepackage{amssymb}
\usepackage{amsmath}
\usepackage{centernot}
\usepackage{scalerel}
\usepackage{stackengine}
\usepackage{xcolor}
\newcommand\showdiv[1]{\overline{\smash{\hstretch{.5}{)}\mkern-3.2mu\hstretch{.5}{)}}#1}}
\newcommand\ph[1]{\textcolor{white}{#1}}


\makeatletter
% we use \prefix@<level> only if it is defined
\renewcommand{\@seccntformat}[1]{%
  \ifcsname prefix@#1\endcsname
    \csname prefix@#1\endcsname
  \else
    \csname the#1\endcsname\quad
  \fi}
% define \prefix@section
\newcommand\prefix@section{}
\newcommand\prefix@subsection{}
\makeatother

\begin{document}

\title{Applied Number Theory: Homework 3}
\author{Joshua Dong}
\date{\today}
\maketitle

\section{2.3}
\subsection{a)}
$g$ is a primitive root of $\mathbb{F}_p$.
Then $g$ is a generator of the group.
Then the order of $g$ is $p$, and $g^p = 1$.
Then $g^n$ for $n \in \{0, 1, 2, ..., p-1\}$, $g^n$ is distinct
for different $n$, and $g^n = g^{n + (p-1)}$.
This effectively defines $p$ unique, disjoint equivalence classes
for the elements in $\{g^n | n \in \mathbb{Z} \}$.
\\\\
If $x = a$, a solution to $g^x = h$, then where $a$ is in the equivalence
class $A$ (since equivialence classes are disjoint and cover the space).
If $x = b$, then b must also be in the class $A$, by definition of our
equivialence class.
Since the elements of $A$ may be expressed as $A = \{a + k(p - 1) | k \in \mathbb{Z}\}$.
Then $p - 1 | a - b$.

\subsection{b)}
Let $a,b,c \in \mathbb{F}^*_p$ such that
$g^a = h_1h_2$, $g^b = h_1$, $g^c = h_2$ for any given $h_1, h_2 \in \mathbb{F}^*_p$.
\\\\
Then $g^bg^c = h_1h_2$.
Then $g^{b+c} = h_1h_2$.
Then $g^{b+c} = g^a$, $a = b+c$ as could be clearly proven by a
uniqueness argument for the equivalence classes in $\mathbb{F}^*_p$.
\\\\
$a = \log_g h_1h_2$, $b = \log_g h_1$, $c = \log_g h_2$ by definition of logarithm.
Then $\log_g h_1h_2 = \log_g h_1 + \log_g h_2$, which is what we sought to show.

\subsection{c)}
Let $a,b \in \mathbb{F}^*_p$ such that
$g^a = h^n$, $g^b = h$ for any given $h \in \mathbb{F}^*_p, n \in \mathbb{Z}$.
\\\\
Then $g^{nb} = (g^{b})^n = h^n$. Then $a = nb$.
\\\\
$a = \log_g h^n$, $b = \log_g h$ by definition of logarithm.
Then $\log_g h^n = n\log_g h$, which is what we sought to show.

\section{2.24}
\subsection{a)}
It is given that $b^2 - a = np$ for some $n \in \mathbb{Z}$, and $p$ is odd and does not divide $b$.
\\*
Since $p$ is odd and does not divide $b$,
we can find a $k \in \mathbb{Z}$ such that
$k \equiv (2b)^{-1}(-n) \mod{p}$ for some given $n \in \mathbb{Z}$.
\\*
Then there exists a $k \in \mathbb{Z}$ such that
$(n + 2bk) \equiv 0 \mod{p}$ for some given $n \in \mathbb{Z}$.
\\*
Then for these $k, n$,
there exists a $m \in \mathbb{Z}$ such that $p(n + 2bk) = p(mp)$.
\\*
Then $0 \equiv np + 2bk \equiv (b^2 - a) + 2bkp + (kp)^2
\equiv p(b + kp)^2 - a \mod{p^2}$.
\\*
Then $a \equiv (b + kp)^2 \mod{p^2}$, which is what we sought to show.

\subsection{b)}
309086

\subsection{c)}
$b^2 - a \equiv 0 \mod{p^n}$.
\\*
Since $2b$ is even and $p$ is odd and does not divide $b$,
$2b$ is invertable.
\\*
Then there exists $j \in \mathbb{Z}$ such that
$j \equiv (-k)(2b)^{-1} \mod{p}$ where
$k(p^n) = b^2 - a$.
\\*
Then there exists $j \in \mathbb{Z}$ such that
$2bj + k \equiv 0 \mod{p}$, $(2bj + k)p^n \equiv 0 \mod{p^{n+1}}$.
\\*
Then there exists $j \in \mathbb{Z}$ such that
$(b^2 - a) + (jp^n)^2 + (2b)jp^n \equiv 0 \mod{p^{n+1}}$.
\\*
Then there exists $j \in \mathbb{Z}$ such that
$b^2 + (jp^n)^2 + (2b)jp^n \equiv a \mod{p^{n+1}}$.
\\*
Then there exists $j \in \mathbb{Z}$ such that
$(b + jp^n)^2 \equiv a \mod{p^{n+1}}$, which is what we sought to show.

\subsection{d)}
We can apply the principle of mathematical induction.
Let $P(n) := $ If $p$ is an odd prime and if $a$ has a square root
modulo $p^n$, then a has a square root modulo $p^n+1$. We already
have shown $P(1)$ and $P(n) \rightarrow P(n+1)$. Then $P(n)$ is
true for all integers $n$.
\\\\
One of the core assumptions was that $p$ is odd, since this
guarantees invertability for $2b$. Thus, $P(n)$ does not hold
if the prime used is 2.

\subsection{e)}
1075 as well as 1122 are the square roots of 3 modulo $13^3$, since 4 is also a square root of 3 modulo 13.

\section{2.27}
Let $G$ be a group with order $\varphi(p) = q_1q_2$,
where $p$ is a prime number.
Let $g \in G$, en element of order $N$.
$N$ can be factored into a product of primes as
$N = q_1 \cdot q_2$.
\\\\
Then we can break the problem of the discrete logorithm of $h = g^x
\mod{p}$ (given some arbitrary $h$) into the smaller problems of
discrete logarithm modulo $q_1$ and $q_2$.
\\\\
By the Chinese Remainder Theorem, $x$ can be retrieved by
way of combining the solutions of the modular equations
$h = g^{y_1} \mod{q_1}$ and $h = g^{y_2} \mod{q_2}$.
\\\\
Suppose we have:
\\
$x = y_1 + q_1z_1$ for some $z_1 \in \mathbb{Z}$ and
$x = y_2 + q_2z_2$ for some $z_2 \in \mathbb{Z}$.
\\\\
Then $(g^x)^{q_2} = (g^{y_1 + q_1z_1})^{q_2}, (g^x)^{q_1} = (g^{y_2 + q_2z_2})^{q_1}$,
\\
$q_1, q_2$ are coprime. Then by Bezout's theorem, there exists $c_1, c_2$
such that $q_1c_1 + q_2c_2 = 1$.
\\\\
Since $q_2x \equiv q_2\log_g{(h)} (\mod{N}),
q_1x \equiv q_1\log_g{(h)} (\mod{N})$.
\\
Then $(q_2c_2 + q_1c_1)x \equiv (q_2c_1 + q_1c_2) \log_g{(h)} (\mod{N})$.
\\
Then $x \equiv \log_g{(h)} (\mod{N})$, which is what we sought to show.
\\\\
Now the discrete logarithm problem has been broken up into two much smaller discrete logarithm problems.

\end{document}
